% Options for packages loaded elsewhere
\PassOptionsToPackage{unicode}{hyperref}
\PassOptionsToPackage{hyphens}{url}
%
\documentclass[
]{article}
\usepackage{amsmath,amssymb}
\usepackage{lmodern}
\usepackage{ifxetex,ifluatex}
\ifnum 0\ifxetex 1\fi\ifluatex 1\fi=0 % if pdftex
  \usepackage[T1]{fontenc}
  \usepackage[utf8]{inputenc}
  \usepackage{textcomp} % provide euro and other symbols
\else % if luatex or xetex
  \usepackage{unicode-math}
  \defaultfontfeatures{Scale=MatchLowercase}
  \defaultfontfeatures[\rmfamily]{Ligatures=TeX,Scale=1}
\fi
% Use upquote if available, for straight quotes in verbatim environments
\IfFileExists{upquote.sty}{\usepackage{upquote}}{}
\IfFileExists{microtype.sty}{% use microtype if available
  \usepackage[]{microtype}
  \UseMicrotypeSet[protrusion]{basicmath} % disable protrusion for tt fonts
}{}
\makeatletter
\@ifundefined{KOMAClassName}{% if non-KOMA class
  \IfFileExists{parskip.sty}{%
    \usepackage{parskip}
  }{% else
    \setlength{\parindent}{0pt}
    \setlength{\parskip}{6pt plus 2pt minus 1pt}}
}{% if KOMA class
  \KOMAoptions{parskip=half}}
\makeatother
\usepackage{xcolor}
\IfFileExists{xurl.sty}{\usepackage{xurl}}{} % add URL line breaks if available
\IfFileExists{bookmark.sty}{\usepackage{bookmark}}{\usepackage{hyperref}}
\hypersetup{
  pdftitle={Analisis Descriptivo},
  hidelinks,
  pdfcreator={LaTeX via pandoc}}
\urlstyle{same} % disable monospaced font for URLs
\usepackage[margin=1in]{geometry}
\usepackage{color}
\usepackage{fancyvrb}
\newcommand{\VerbBar}{|}
\newcommand{\VERB}{\Verb[commandchars=\\\{\}]}
\DefineVerbatimEnvironment{Highlighting}{Verbatim}{commandchars=\\\{\}}
% Add ',fontsize=\small' for more characters per line
\usepackage{framed}
\definecolor{shadecolor}{RGB}{248,248,248}
\newenvironment{Shaded}{\begin{snugshade}}{\end{snugshade}}
\newcommand{\AlertTok}[1]{\textcolor[rgb]{0.94,0.16,0.16}{#1}}
\newcommand{\AnnotationTok}[1]{\textcolor[rgb]{0.56,0.35,0.01}{\textbf{\textit{#1}}}}
\newcommand{\AttributeTok}[1]{\textcolor[rgb]{0.77,0.63,0.00}{#1}}
\newcommand{\BaseNTok}[1]{\textcolor[rgb]{0.00,0.00,0.81}{#1}}
\newcommand{\BuiltInTok}[1]{#1}
\newcommand{\CharTok}[1]{\textcolor[rgb]{0.31,0.60,0.02}{#1}}
\newcommand{\CommentTok}[1]{\textcolor[rgb]{0.56,0.35,0.01}{\textit{#1}}}
\newcommand{\CommentVarTok}[1]{\textcolor[rgb]{0.56,0.35,0.01}{\textbf{\textit{#1}}}}
\newcommand{\ConstantTok}[1]{\textcolor[rgb]{0.00,0.00,0.00}{#1}}
\newcommand{\ControlFlowTok}[1]{\textcolor[rgb]{0.13,0.29,0.53}{\textbf{#1}}}
\newcommand{\DataTypeTok}[1]{\textcolor[rgb]{0.13,0.29,0.53}{#1}}
\newcommand{\DecValTok}[1]{\textcolor[rgb]{0.00,0.00,0.81}{#1}}
\newcommand{\DocumentationTok}[1]{\textcolor[rgb]{0.56,0.35,0.01}{\textbf{\textit{#1}}}}
\newcommand{\ErrorTok}[1]{\textcolor[rgb]{0.64,0.00,0.00}{\textbf{#1}}}
\newcommand{\ExtensionTok}[1]{#1}
\newcommand{\FloatTok}[1]{\textcolor[rgb]{0.00,0.00,0.81}{#1}}
\newcommand{\FunctionTok}[1]{\textcolor[rgb]{0.00,0.00,0.00}{#1}}
\newcommand{\ImportTok}[1]{#1}
\newcommand{\InformationTok}[1]{\textcolor[rgb]{0.56,0.35,0.01}{\textbf{\textit{#1}}}}
\newcommand{\KeywordTok}[1]{\textcolor[rgb]{0.13,0.29,0.53}{\textbf{#1}}}
\newcommand{\NormalTok}[1]{#1}
\newcommand{\OperatorTok}[1]{\textcolor[rgb]{0.81,0.36,0.00}{\textbf{#1}}}
\newcommand{\OtherTok}[1]{\textcolor[rgb]{0.56,0.35,0.01}{#1}}
\newcommand{\PreprocessorTok}[1]{\textcolor[rgb]{0.56,0.35,0.01}{\textit{#1}}}
\newcommand{\RegionMarkerTok}[1]{#1}
\newcommand{\SpecialCharTok}[1]{\textcolor[rgb]{0.00,0.00,0.00}{#1}}
\newcommand{\SpecialStringTok}[1]{\textcolor[rgb]{0.31,0.60,0.02}{#1}}
\newcommand{\StringTok}[1]{\textcolor[rgb]{0.31,0.60,0.02}{#1}}
\newcommand{\VariableTok}[1]{\textcolor[rgb]{0.00,0.00,0.00}{#1}}
\newcommand{\VerbatimStringTok}[1]{\textcolor[rgb]{0.31,0.60,0.02}{#1}}
\newcommand{\WarningTok}[1]{\textcolor[rgb]{0.56,0.35,0.01}{\textbf{\textit{#1}}}}
\usepackage{graphicx}
\makeatletter
\def\maxwidth{\ifdim\Gin@nat@width>\linewidth\linewidth\else\Gin@nat@width\fi}
\def\maxheight{\ifdim\Gin@nat@height>\textheight\textheight\else\Gin@nat@height\fi}
\makeatother
% Scale images if necessary, so that they will not overflow the page
% margins by default, and it is still possible to overwrite the defaults
% using explicit options in \includegraphics[width, height, ...]{}
\setkeys{Gin}{width=\maxwidth,height=\maxheight,keepaspectratio}
% Set default figure placement to htbp
\makeatletter
\def\fps@figure{htbp}
\makeatother
\setlength{\emergencystretch}{3em} % prevent overfull lines
\providecommand{\tightlist}{%
  \setlength{\itemsep}{0pt}\setlength{\parskip}{0pt}}
\setcounter{secnumdepth}{-\maxdimen} % remove section numbering
\ifluatex
  \usepackage{selnolig}  % disable illegal ligatures
\fi

\title{Analisis Descriptivo}
\author{}
\date{\vspace{-2.5em}}

\begin{document}
\maketitle

\hypertarget{r-markdown}{%
\subsection{R Markdown}\label{r-markdown}}

\hypertarget{section}{%
\subsection{}\label{section}}

Datos en cuestión

\begin{Shaded}
\begin{Highlighting}[]
\NormalTok{project\_data }\OtherTok{\textless{}{-}} \FunctionTok{read.delim}\NormalTok{(}\StringTok{"broadway{-}shows.txt"}\NormalTok{)}
\FunctionTok{str}\NormalTok{(project\_data)}
\end{Highlighting}
\end{Shaded}

\begin{verbatim}
## 'data.frame':    33 obs. of  8 variables:
##  $ Season         : int  1984 1985 1986 1987 1988 1989 1990 1991 1992 1993 ...
##  $ Gross..M.      : num  209 190 208 253 262 282 267 293 328 356 ...
##  $ Attendance     : num  7.26 6.54 7.04 8.14 7.96 8.04 7.32 7.38 7.86 8.11 ...
##  $ Playing.weeks  : int  1078 1041 1039 1113 1108 1070 976 905 1019 1066 ...
##  $ New.Productions: int  33 34 41 30 33 39 30 37 34 41 ...
##  $ Mean.ticket    : num  28.8 29.1 29.5 31.1 32.9 ...
##  $ Pct.sold       : num  0.0471 0.044 0.0474 0.0512 0.0503 ...
##  $ LogGross       : num  2.32 2.28 2.32 2.4 2.42 ...
\end{verbatim}

\hypertarget{informaciuxf3n-de-recaudo}{%
\subsubsection{Información de recaudo}\label{informaciuxf3n-de-recaudo}}

\begin{Shaded}
\begin{Highlighting}[]
\FunctionTok{summary}\NormalTok{(project\_data}\SpecialCharTok{$}\NormalTok{Gross..M.)}
\end{Highlighting}
\end{Shaded}

\begin{verbatim}
##    Min. 1st Qu.  Median    Mean 3rd Qu.    Max. 
##   190.0   328.0   643.0   691.7   943.0  1449.0
\end{verbatim}

De aquí resulta inmediato que los datos están más dispersos entre los
dos primeros cuartiles, y que los valores máximos y mínimos se alejan
considerablemente de la media, pero también de los cuartiles.

\begin{Shaded}
\begin{Highlighting}[]
\FunctionTok{par}\NormalTok{(}\AttributeTok{mfrow=} \FunctionTok{c}\NormalTok{(}\DecValTok{1}\NormalTok{,}\DecValTok{2}\NormalTok{))}
\NormalTok{recaudo\_hist }\OtherTok{\textless{}{-}} \FunctionTok{hist}\NormalTok{(project\_data}\SpecialCharTok{$}\NormalTok{Gross..M.,}\AttributeTok{main=}\StringTok{"Histograma de recaudo"}\NormalTok{,}\AttributeTok{xlab=}\StringTok{"Recaudo (millones de $)"}\NormalTok{,}\AttributeTok{ylab=}\StringTok{"frecuencia"}\NormalTok{)}
\NormalTok{recaudo\_bp }\OtherTok{\textless{}{-}} \FunctionTok{boxplot}\NormalTok{(project\_data}\SpecialCharTok{$}\NormalTok{Gross..M.,}\AttributeTok{main=}\StringTok{"Diagrama de caja de recaudo"}\NormalTok{,}\AttributeTok{ylab=}\StringTok{"recaudo (millones de $"}\NormalTok{)}
\end{Highlighting}
\end{Shaded}

\includegraphics{test_files/figure-latex/unnamed-chunk-3-1.pdf}

Como se observo del resumen anterior, la dispersión en los datos entre
los cuartiles extremos es casi la misma.

Si comparamos los valores extremos con los años en que se produjeron,
vemos que

\begin{Shaded}
\begin{Highlighting}[]
\NormalTok{project\_data}\SpecialCharTok{$}\NormalTok{Season}
\end{Highlighting}
\end{Shaded}

\begin{verbatim}
##  [1] 1984 1985 1986 1987 1988 1989 1990 1991 1992 1993 1994 1995 1996 1997 1998
## [16] 1999 2000 2001 2002 2003 2004 2005 2006 2007 2008 2009 2010 2011 2012 2013
## [31] 2014 2015 2016
\end{verbatim}

\begin{Shaded}
\begin{Highlighting}[]
\NormalTok{project\_data}\SpecialCharTok{$}\NormalTok{Gross..M.}
\end{Highlighting}
\end{Shaded}

\begin{verbatim}
##  [1]  209  190  208  253  262  282  267  293  328  356  406  436  499  558  588
## [16]  603  666  643  721  771  769  862  939  938  943 1020 1081 1139 1139 1269
## [31] 1365 1373 1449
\end{verbatim}

Corresponden a la recaudación más antigua y más reciente
respectivamente. Observemos entonces la evolución de la recaudación a lo
largo de los años.

\begin{Shaded}
\begin{Highlighting}[]
\FunctionTok{plot}\NormalTok{(}\AttributeTok{main=}\StringTok{"Evolución de la recaudación en Broadway para los años (1984{-}2016)"}\NormalTok{,project\_data}\SpecialCharTok{$}\NormalTok{Season,project\_data}\SpecialCharTok{$}\NormalTok{Gross..M.,}\AttributeTok{xlab=}\StringTok{"Año"}\NormalTok{,}\AttributeTok{ylab=}\StringTok{"Recaudación"}\NormalTok{)}
\end{Highlighting}
\end{Shaded}

\includegraphics{test_files/figure-latex/unnamed-chunk-5-1.pdf} Este
crecimiento consistente a través de los años parece explicar la simetría
mostrada entre los cuartiles extremos y la media, pues para los valores
de recaudo asociados al rango intercuartil, no hay una tendencia
concreta para los recaudos mas que el aumento y (en pocas ocasiones)
descenso en los mismos.

\hypertarget{informaciuxf3n-de-concurrencia}{%
\subsection{Información de
concurrencia}\label{informaciuxf3n-de-concurrencia}}

\hypertarget{semanas-activas-para-las-obras}{%
\subsection{Semanas activas para las
obras}\label{semanas-activas-para-las-obras}}

\hypertarget{nuxfamero-de-obras-nuevas}{%
\subsection{Número de obras nuevas}\label{nuxfamero-de-obras-nuevas}}

\hypertarget{precio-promedio-de-los-tickets}{%
\subsection{Precio promedio de los
tickets}\label{precio-promedio-de-los-tickets}}

\hypertarget{porcentaje-de-tickets-que-no-se-venden}{%
\subsection{Porcentaje de tickets que no se
venden}\label{porcentaje-de-tickets-que-no-se-venden}}

\hypertarget{logaritmo-de-lo-recaudado.}{%
\subsection{Logaritmo de lo
recaudado.}\label{logaritmo-de-lo-recaudado.}}

\end{document}
